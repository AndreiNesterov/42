% This is samplepaper.tex, a sample chapter demonstrating the
% LLNCS macro package for Springer Computer Science proceedings;
% Version 2.20 of 2017/10/04
%
\documentclass[runningheads]{llncs}
%
\usepackage{graphicx}
\usepackage{todonotes}
% Used for displaying a sample figure. If possible, figure files should
% be included in EPS format.
%
% If you use the hyperref package, please uncomment the following line
% to display URLs in blue roman font according to Springer's eBook style:
% \renewcommand\UrlFont{\color{blue}\rmfamily}

\usepackage[backend=biber,sorting=nyt,bibencoding=latin1,abbreviate=false,mincrossrefs=3,style=numeric,maxnames=30]{biblatex}
\usepackage{multirow}
\addbibresource{bib/db.bib}


\begin{document}
%
\title{Learning Class Definitions by Link Prediction (preliminary)\thanks{Technical report of the task force 42 from ISWS 2022 led by Heiko Paulheim.}}

%
%\titlerunning{Abbreviated paper title}
% If the paper title is too long for the running head, you can set
% an abbreviated paper title here
%
\author{First Author\inst{1}\orcidID{0000-1111-2222-3333} \and
Second Author\inst{2,3}\orcidID{1111-2222-3333-4444} \and
Third Author\inst{3}\orcidID{2222--3333-4444-5555}}
%
\authorrunning{F. Author et al.}
% First names are abbreviated in the running head.
% If there are more than two authors, 'et al.' is used.
%
\institute{Princeton University, Princeton NJ 08544, USA \and
Springer Heidelberg, Tiergartenstr. 17, 69121 Heidelberg, Germany
\email{lncs@springer.com}\\
\url{http://www.springer.com/gp/computer-science/lncs} \and
ABC Institute, Rupert-Karls-University Heidelberg, Heidelberg, Germany\\
\email{\{abc,lncs\}@uni-heidelberg.de}}
%
\maketitle              % typeset the header of the contribution
%
\begin{abstract}
The abstract should briefly summarize the contents of the paper in
15--250 words.  Test citation \cite{Berners-Lee2001}

\keywords{First keyword  \and Second keyword \and Another keyword.}
\end{abstract}

\section{Research Questions}
\label{sec:rq}
This technical report aims to investigate whether knowledge graph embeddings are helpful in entity class identification in DBpedia by developing an experimental set-up. To answer this question, we calculate the link prediction scores for entities in pre-selected classes and compare them to the frequency-based baselines.


\section{Empirical Semantics}
\label{sec:def}
The empirical semantics perspective of this work considers classes and properties of the DBpedia ontology. The main problematic aspect is the semantic expressivity and formalisation of the DBpedia categories (\textit{skos:Concept}) derived from Wikipedia.

\section{Introduction}
\label{sec:intro}
\textit{[1 Page]}

\noindent
Explain your perspective on the problem of Empirical Semantics. 
Give both the intuition and motivate, by relying on use cases and examples, why this perspective is important. 
Briefly describe what is the state of the art and how you’re pushing it with your contribution. 
Also mention what data and methods you use in your work. 
Conclude by clearly stating what is your contribution.

Problem definition by illustrative example.

Wikipedia categories define topics of resources in DBpedia.
For example, the entity of type 'work' 'The Shining (novel)' \footnote{https://dbpedia.org/page/The\_Shining\_(novel)} in DBpedia is defined through \textit{dct:subject} property with several category URIs derived from Wikipedia, including the category 'Novels by Stephen King'\footnote{https://dbpedia.org/page/Category:Novels\_by\_Stephen\_King}.
This category, which is formally a \textit{skos:Concept}, has such implicit information as type of written work, fiction genre ('Novel') and author ('Stephen King').
However, this implicit human-readable information included in a Wikipedia URI does not bring any additional formal semantics to the resource in DBpedia.
Which properties of the resource 'The Shining (novel)' help automatically reason that it belongs to the category 'Novels by Stephen King'?
What are the common properties of the resources, which are defined through the same Wikipedia category?
Is it possible to detect such properties and assign new classes automatically, so that the nuanced categorisation of the resources can be made machine interpretable?
This empirical semantics observations drive our research question and the experimental set-up, which aims to enrich the DBpedia ontology based on Wikidata categorisation. 

About the method.
Why do we focus on embeddings?
We experiment with knowledge graph embeddings (in particular, RDF2vec) because..

About Related Work.
There is work where embeddings were used to create class hierarchies, but not to predict classes (?)
This is what novel in our direction.

About the data.
For our experiment, we manually picked 13 categories from DBpedia that represent tangible and intangible real-world objects (such as books, persons, languages) and can be easily described with several properties.
We collected all triples contained in the selected categories, taking only triples with DBpedia ontology predicates, since we aim at enriching this ontology.
Additionally, we filtered out triples referring to Wikidata identifiers (such as 'dbo:wikiPageID') and triples with literal values as they are not relevant for our study.
The detailed data overview is presented in Section 5.

Contributions.
Results of the experiment.

\section{Related Work}
\label{sec:related}
\textit{[1 Page]}

The objective of enriching knowledge graphs such as DBpedia, has been attracting the efforts of the scientific community during the last years, and continues in the same direction still nowadays. As this work, in order to attack the issue, highlights the path of exploiting the categories, some of the main related work is presented as follows:
\begin{itemize}
  \item Uncovering the Semantics of Wikipedia Categories. \cite{HeistHeiko2019}
  This work introduces an approach for the discovery of category axioms that uses information from the category network, category instances, and their lexicalisations with DBpedia as background knowledge.
\end{itemize}

\noindent
List the main relevant work (a bullet list is ok) and for each of them write a paragraph describing (i) the key contribution of the related work, (ii) how your contribution relates/differentiate from it.


\section{Resources}
\label{sec:resources}
\textit{[1 page]}

\noindent
Our use-case focusses on DBpedia categories (why?). A category in DBpedia has a type of \textit{skos:Concept}.
There are more than 2 million categories in DBpedia. We manually selected 13 categories (motivation?) for our experiment. For every category, we manually selected predicate-object pairs that represent and distinguish a category. For example, the pair "author"-"Stephen\_King" was chosen as an indicative of the category "Novels\_by\_Stephen\_King". For six categories, more than one predicatex-object pair was selected. 

Three datasets were retrieved from DBpedia with SPARQL-queries:
1. Getting the number of subjects with the same predicate and object for every category.
Criteria: (1) predicates are in the DBpedia ontology, (2) predicates not referring to Wikipedia (not containing the word "wiki" in their names).
For example, there are 44 objects with predicate "author" and object "Stephen\_King" in the category "Novels\_by\_Stephen\_King".
This predicate-object pair is the most frequent in the category and it is indicative for the category.
The purpose of this dataset is to calculate the frequency of the predicate-object pairs in every category (ranking) and calculate reciprocal rank. The overview of the selected categories, their indicative object-predicate pairs and ranks is presented in \ref{tab:dbpedia_categories}. Refer to the SPARQL-query. 
2. Getting all triples for every category. Criteria: (1), (2), and (3) objects with URI containing "http://dbpedia.org/resource/" to filter out objects with literal values.
This dataset is used for evaluation. Refer to SPARQL-query.
3. Getting all combinations of subjects and predicate-object-pairs in a category. Criteria: (1), (2), (3). Refer to SPARQL-query.


\begin{table}[]
\caption{DBpedia categories selected for experiment}
\label{tab:dbpedia_categories}
\begin{tabular}{|l|l|l|l|l|l|}
\hline
Category URI & Predicate & Object & N objects & Rank & Reciprocal rank \\ \hline
Novels\_by\_Stephen\_King & author & Stephen\_King & 44 & 1 &  \\ \hline
\multirow{2}{*}{Swedish\_death\_metal\_musical\_groups} & genre & Death\_metal & 61 &  &  \\ \cline{2-6} 
 & hometown & Sweden & 22 &  &  \\ \hline
Red\_Hot\_Chili\_Peppers\_songs & artist & Red\_Hot\_Chili\_Peppers & 49 &  &  \\ \hline
\multirow{3}{*}{English\_pop\_pianists} & genre & Pop\_music &  &  &  \\ \cline{2-6} 
 & birthPlace & England &  &  &  \\ \cline{2-6} 
 & instrument & Piano &  &  &  \\ \hline
1990s\_American\_sitcoms & genre & Sitcom & 273 &  &  \\ \hline
Films\_produced\_by\_Denzel\_Washington & producer & Denzel\_Washington & 6 &  &  \\ \hline
\multirow{3}{*}{Hilltowns\_in\_Emilia-Romagna} & region & Emilia-Romagna & 2 &  &  \\ \cline{2-6} 
 & country & Italy & 2 &  &  \\ \cline{2-6} 
 & province & Province\_of\_Forlì-Cesena & 1 &  &  \\ \hline
Languages\_of\_Namibia & spokenIn & Namibia & 13 &  &  \\ \hline
Italian\_Renaissance\_painters & movement & Italian\_Renaissance &  &  &  \\ \hline
\multirow{2}{*}{Philippine\_television\_talk\_shows} & genre & Talk\_show & 78 &  &  \\ \cline{2-6} 
 & country & Philippines & 23 &  &  \\ \hline
Scottish\_clans & country & Scotland & 1 &  &  \\ \hline
Argentine\_Primera\_División\_players & team & Argentina\_national\_football\_team &  &  &  \\ \hline
Birds\_of\_Europe & class & Bird & 1 &  &  \\ \hline
\end{tabular}
\end{table}

\section{Proposed approach}
\label{sec:approach}
\textit{[2 pages]}

\noindent
Describe your proposed method. 


\section{Evaluation and Results: Use case/Proof of concept - Experiments}
\label{sec:evaluation}
\textit{[2 pages]}

\noindent
Show here that your proposed approach addresses your research questions or how you intend to show it. This can be done by either or both:

\begin{itemize}
    \item Describing an experimental setting design, including research hypotheses, methods and metrics of measurements 

    \item Describing a proof of concept/use case, based on real data, that support your claim
\end{itemize}



\section{Discussion and Conclusions}
\label{sec:conclusions}
\textit{[1 page]}

\noindent
Identify strengths and weaknesses of your proposal, discuss lessons learned: what are the key issues you have encountered or that you think should be taken into account to develop your proposal/experiments, and what are possible ways to address them. 





\printbibliography
\end{document}