\textit{[1 Page]}

\noindent
Explain your perspective on the problem of Empirical Semantics. 
Give both the intuition and motivate, by relying on use cases and examples, why this perspective is important. 
Briefly describe what is the state of the art and how you’re pushing it with your contribution. 
Also mention what data and methods you use in your work. 
Conclude by clearly stating what is your contribution.

Problem definition by illustrative example.

Wikipedia categories define topics of resources in DBpedia.
For example, the entity of type 'work' 'The Shining (novel)' \footnote{https://dbpedia.org/page/The\_Shining\_(novel)} in DBpedia is defined through \textit{dct:subject} property with several category URIs derived from Wikipedia, including the category 'Novels by Stephen King'\footnote{https://dbpedia.org/page/Category:Novels\_by\_Stephen\_King}.
This category, which is formally a \textit{skos:Concept}, has such implicit information as type of written work, fiction genre ('Novel') and author ('Stephen King').
However, this implicit human-readable information included in a Wikipedia URI does not bring any additional formal semantics to the resource in DBpedia.
Which properties of the resource 'The Shining (novel)' help automatically reason that it belongs to the category 'Novels by Stephen King'?
What are the common properties of the resources, which are defined through the same Wikipedia category?
Is it possible to detect such properties and assign new classes automatically, so that the nuanced categorisation of the resources can be made machine interpretable?
This empirical semantics observations drive our research question and the experimental set-up, which aims to enrich the DBpedia ontology based on Wikidata categorisation. 

About the method.
Why do we focus on embeddings?
We experiment with knowledge graph embeddings (in particular, RDF2vec) because..

About Related Work.
There is work where embeddings were used to create class hierarchies, but not to predict classes (?)
This is what novel in our direction.

About the data.
For our experiment, we manually picked 13 categories from DBpedia that represent tangible and intangible real-world objects (such as books, persons, languages) and can be easily described with several properties.
We collected all triples contained in the selected categories, taking only triples with DBpedia ontology predicates, since we aim at enriching this ontology.
Additionally, we filtered out triples referring to Wikidata identifiers (such as 'dbo:wikiPageID') and triples with literal values as they are not relevant for our study.
The detailed data overview is presented in Section 5.

Contributions.
Results of the experiment.